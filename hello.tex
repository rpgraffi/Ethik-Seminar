%!TEX program = xelatex
\documentclass{article}
\begin{document}
\section*{Possible topics:}


\section*{Legitimität und demokratische Prinzipien}
\begin{itemize}
    \item Volkssouveränität: Wie kann sichergestellt werden, dass die Entscheidungen der KI im Einklang mit dem Willen des Volkes stehen, wenn die KI die Politiker ersetzt?
    \item Gewaltenteilung: Wie lässt sich das Prinzip der Gewaltenteilung aufrechterhalten, wenn die KI legislative, exekutive und möglicherweise sogar judikative Funktionen übernimmt?
    \item Partizipation und Repräsentation: Wie können Bürger in einem solchen System ihre Interessen artikulieren und Einfluss auf die Entscheidungen der KI nehmen?
\end{itemize}

\section*{Ethische Herausforderungen}
\begin{itemize}
    \item Objektivität und Neutralität: Wie kann gewährleistet werden, dass die KI objektive und neutrale Entscheidungen trifft, die nicht von voreingenommenen Daten oder Algorithmen beeinflusst werden?
    \item Verantwortlichkeit: Wer ist für die Entscheidungen der KI verantwortlich, wenn Fehler passieren oder negative Konsequenzen eintreten?
    \item Transparenz und Nachvollziehbarkeit: Wie können die Entscheidungen der KI für die Bürger transparent und nachvollziehbar gemacht werden?
\end{itemize}

\section*{Gesellschaftliche Auswirkungen}
\begin{itemize}
    \item Vertrauen in die Regierung: Wie wirkt sich der Einsatz von KI auf das Vertrauen der Bürger in die Regierung und die demokratischen Institutionen aus?
    \item Soziale Ungleichheit: Könnte der Einsatz von KI bestehende soziale Ungleichheiten verstärken oder neue Formen der Diskriminierung schaffen?
    \item Machtverteilung: Wie verändert sich die Machtverteilung in der Gesellschaft, wenn KI eine zentrale Rolle in der Politik übernimmt?
\end{itemize}

\end{document}
